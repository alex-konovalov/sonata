%
\Chapter{Planar nearrings}
%

We recall the definition of planar nearrings and basic results 
(see \cite{Clay:Nearrings}).
Let $(N,+,\cdot)$ be a left nearring. For $a,b \in N$ we define $a \equiv b$
iff $a\cdot n = b\cdot n$ for all $n\in N$. If $a \equiv b$, then $a$ and $b$
are called *equivalent multipliers*.
A nearring $N$ is called *planar* if $| N/_{\equiv} | \ge 3$ and if 
for any two non-equivalent multipliers $a$ and $b$ in $N$, for any $c\in N$, 
the equation $a\cdot x = b\cdot x + c$ has a unique solution. 

A *Ferrero pair* is a pair of finite groups $(N,\Phi)$ such that $\Phi$
is a fixed-point-free automorphism group of $(N,+)$.   

Starting with a Ferrero pair $(N,\Phi)$ we can construct a planar nearring
in the following way:
Select representatives, say $e_{1},\ldots,e_{t}$, for some or all of the
non-trivial orbits of $N$ under $\Phi$. 
Let $C = \Phi(e_1)\cup\ldots\cup\Phi(e_t)$.
For each $x\in N$ we define $a\cdot x = 0$ for $a\in N\setminus C$, and 
$a\cdot x=\phi_{a}(x)$ for $a\in\Phi(e_{i})\subset C$ and $\phi_{a}(e_{i})=a$.
Then $(N,+,\cdot)$ is a (left) planar nearring with $|N/_{\equiv}| = |\Phi|+1$.

Every finite planar nearring can be constructed from some Ferrero pair 
together with a set of orbit representatives in this way.

*The problem:* Find a planar nearring with 25 elements and 9 pairwise
non-equivalent multipliers.

*The solution:* We follow the Ferrero method described above for defining a
nearring multiplication on an additive group. 
First we have to find a fixed-point-free (fpf) automorphism group of
order $8$ on a group of order $25$. %and choose some orbit representatives.

We start with the cyclic group of order $25$:
First of all we ask for the existence of an fpf automorphism group
on `CyclicGroup(25)' by computing an upper bound for its order.

\beginexample 
    gap> FpfAutomorphismGroupsMaxSize( CyclicGroup(25) );
    [ 4, 1 ]
\endexample

This function returns a list with two integers, $4$ and $1$. 
The first number is an upper bound for the size of an fpf automorphism group;
if there is a metacyclic fpf automorphism group, then it has a cyclic normal
subgroup of index dividing the second number. These bounds are not sharp.
%
If the upper bound for the size of an fpf automorphism group on some group
is $1$, we know that there is no nontrivial fpf automorphism group, no 
Ferrero pair, and no planar nearring on this group at all.

Here, SONATA does not exclude the possibility that the cyclic group of
order $25$ has an fpf automorphism group of order $4$. 
However, we can be sure that all fpf automorphism groups are cyclic and that
none of them has size $8$.

Thus we have to consider the elementary abelian group of order 25 instead.

\beginexample
    gap> FpfAutomorphismGroupsMaxSize( ElementaryAbelianGroup(25) );
    [ 24, 2 ]
\endexample

There might even exist an fpf automorphism group of order $24$. (In fact
there is more than one. The reference manual explains how to obtain all
nearfields of size $25$.) 
For our example, we could compute either a cyclic automorphism group or one 
isomorphic to the quaternion group with 8 elements. Let's try the latter.
 
\beginexample
    gap> aux := FpfAutomorphismGroupsMetacyclic( [5,5], 4, -1 );
    [ [ [ [ f1, f2 ] -> [ f1^2, f2^3 ], [ f1, f2 ] -> [ f2^4, f1 ] ] ], 
      <pc group of size 25 with 2 generators> ]
\endexample

Here, the function `FpfAutomorphismGroupsMetacyclic' determines the metacyclic
fpf automorphism groups on `AbelianGroup([5,5])' with generators $p,q$
satisfying $p^4 = 1, p^q = p^{-1}$, and $q^2 = p^2$. For each conjugacy class
of such groups one representative is given. Conjugacy is determined within the
whole automorphism group of `AbelianGroup([5,5])'. 
The actual output of the function is a list with 2 elements.
The first is not the list of fpf groups up to conjugacy but the list of
automorphisms $p,q$ generating those groups.
The second element is simply the group `AbelianGroup([5,5])', on which the
automorphisms act.

Since there is only one pair of generators $p,q$, all fpf automorphism groups
isomorphic to the quaternion group are conjugate. Now, we have our Ferrero
pair $(G, \Phi)$.

\beginexample
    gap> phi := Group( aux[1][1] );
    <group with 2 generators>
    gap> G := aux[2];         
    <pc group of size 25 with 2 generators>
\endexample

%By picking any one non-zero element of $G$ as representative we can now
%define a planar nearring according to the Ferrero method.
Next we have to pick some orbit representatives. 
We note that for a fixed Ferrero pair distinct choices of representatives 
may yield isomorphic nearrings. The function
`OrbitRepresentativesForPlanarNearRing' returns exactly one set of
representatives of given cardinality for each isomorphism class of planar
nearrings which can be generated from $(G, \Phi)$.

\beginexample
    gap> OrbitRepresentativesForPlanarNearRing( G, phi, 1 );
    [ [ f1 ] ]
\endexample

This tells us that all planar nearrings obtained from $(G,\Phi)$ with one
orbit representative are in fact isomorphic. 
What happens if we choose $2$ representatives?

\beginexample
    gap> reps := OrbitRepresentativesForPlanarNearRing( G, phi, 2 );
    [ [ f1, f1*f2 ], [ f1, f1^2*f2^3 ] ]
\endexample

We obtain $2$ non-isomorphic planar near-rings. Let's just construct one of
them. The result will be an `ExplicitMultiplicationNearRing'.

\beginexample
    gap> n := PlanarNearRing( G, phi, reps[1] );
    ExplicitMultiplicationNearRing ( <pc group of size 25 with 
    2 generators> , multiplication )
\endexample

How many non-isomorphic planar nearrings can be defined from our Ferrero pair
$(G,\Phi)$ in total? Since there are $3$ non-trivial orbits of $\Phi$ on $G$,
we may choose up to $3$ representatives.
 
\beginexample
    gap> Length(OrbitRepresentativesForPlanarNearRing( G, phi, 3 ));
    6
\endexample

Summing all up, we get exactly $9$ non-isomorphic planar nearrings with 
elementary abelian additive group of order $25$ whose multiplication is
defined using a quaternion group of fpf automorphisms. 



%%% Local Variables: 
%%% mode: latex
%%% TeX-master: "manual"
%%% End: 
